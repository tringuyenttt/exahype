\documentclass[a4paper]{article}

\usepackage[english]{babel}
\usepackage[utf8]{inputenc}
\usepackage{amsmath}

\title{Gravitating Hydrodynamics in ExaHyPE}
\author{Sven Köppel}
\date{\today}

\begin{document}
\maketitle

This document describes a small benchmark/demonstrator
application which implements the classical hydro equations
in an external potential. They are hyperbolically given
as
%
\begin{alignat}{2}
\partial_t \rho + \nabla ( \rho \vec v ) &= 0
\\
\partial_t (\rho \vec v) + \nabla ( \rho \vec v^2 + P)
&= - \rho \nabla \Phi
\\
\partial_t (\rho E) + \nabla ((\rho E + P)\vec v)
&= - \rho \vec v \cdot \nabla \Phi
\end{alignat}
%
with the fluid density $\rho$, fluid velocity $\vec v$,
isotropic fluid pressure $P$,
total energy $E=\vec v^2 / 2 + \epsilon$ per unit mass,
internal energy $\epsilon$ per unit mass, the spatial divergence or gradient vector $\nabla$ and time $t$. Employing
the one-parameteric ideal equation of state reads
$P=(\gamma-1)\rho \epsilon$ with some fixed
specific heat ratio $\gamma$.

In order to describe a self-gravitating fluid, we would need
to employ a possion solver to evolve
\begin{equation}
\Phi(\vec r) = G \int \mathrm{d}^3 x
\frac{\rho(\vec x)}{\left| \vec r - \vec x \right|}.
\end{equation}
by means of
\begin{equation}
\nabla^2 \Phi = 4\pi G \rho
\end{equation}
However, since ExaHyPE restricts to hyperbolic PDEs, for this
demonstrator we keep instead the gravitational potential fixed.
$\nabla \Phi$ acts then as a static source to the hyperbolic
evolution system.



\end{document}